\epigraph{\textit{Derivatives are financial weapons of mass destruction.}}{––\textsc{Warren Buffett}, Billionaire Investor}

\section*{}

Bank accounts. Credit Cards. Loans. Mortgages. These products have one thing in common: they are all related to interest rates. Of course, that is not to say they are the same. A savings account offering 30\% interest may be wishful thinking, but on a credit card, that is fairly standard. The important thing is that interest rates are everywhere. 

As we transition from the world of retail to wholesale finance, interest rates become no less important––only, the sums become much larger. Such vast amounts of money introduce much greater levels of uncertainty and a keen desire to manage risk. Enter derivatives. Specifically interest rate derivatives. The \$500 trillion market to help companies manage their interest rate risks \citep{BIS_Ch2_1}.

One of the most popular products is interest rate swaps. Since their introduction in the 1980's they have become a key tool for those looking to convert variable, floating rate exposure into known fixed rate payments. Their theoretical pricing amounts to no more than straightforward time value of money calculations, using an appropriate discounting rate, and has been well known since their inception. Most financial textbooks with a section on interest rate products will include at least some discussion of interest rate swaps \citep{sadr2009interest}  \citep{flavell2012swaps} \citep{wilmott2013paul} \citep{veronesi2016handbook}.

In recent years, however, the previously illusive world of cryptocurrencies has exploded in popularity. Despite its volatility, in part due its sensitivity to an army of fearless (or perhaps reckless) retail traders and celebrity endorsements such as serial tweeter Elon Musk, it is clear that the cryptocurrency market is here to stay. 

This dramatic rise in attention has been coupled with increased academic focus, yet current research tends to focus on the creation, mechanics, and sustainability of existing cryptocurrencies and their associated blockchain protocols, with a little reserved for the analysis and extension of traditional products and derivatives to the world of decentralised finance. 

Companies at the forefront of new cryptocurrency products have begun to consider the world of interest rate derivatives––and specifically interest rate swaps––as an opportunity to extend the traditional case to the crypto market. Of those few products that exist though, there is little analysis or research specifically devoted to them, and even fewer which provide a coherent bridge between traditional interest rate swaps and their crypto counterparts.

As a result, we state the primary objective of this project: to bridge the gap between the established literature of interest rates and its derivatives and the emerging world of cryptocurrency derivatives. Rather than considering traditional and decentralised finance as two distinct and separate areas, we take a product focused approach by considering interest rate swaps as a whole, under which we group and review the two markets. This ensures the project is concise and self-contained. This will likely be beneficial for traditional and crypto market participants seeking to understand the role of interest rate swaps and its possible use with crypto, in addition to academics for whom this project may generate interest in further, deeper research into the mathematical and computational aspects of crypto interest rate swaps.

The project is structured as follows. Chapter 2 introduces the background and context to the topic, including some necessary mathematical preliminaries. In this chapter, we also cover the historical and regulatory context of interest rate derivatives and cryptocurrencies. Chapter 3 covers the world of traditional finance and the definition and terminology of various interest rate derivatives. We include the pricing of traditional interest rate swaps; a description of two curve construction methodologies; their Python implementation using market data; and a discussion of the practical considerations for pricing traditional interest rate swaps. Chapter 4 extends the coverage of the previous chapter to cryptocurrencies, and includes the definition and terminology of various cryptocurrency products; a  discussion of a particular crypto interest rate model; the pricing of crypto interest rate swaps; and possible avenues for new interest rate swap-like products within the crypto market. Finally, Chapter 5 ends with some concluding remarks reflecting upon whether we have achieved our primary objective, and offering some thoughts and recommendations on future work. 