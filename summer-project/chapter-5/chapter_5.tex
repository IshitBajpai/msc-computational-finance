\epigraph{\textit{In each revolution, we create a brand new way of trading, transacting and storing value––but we don’t get rid of the old ones.}}{––\textsc{Chris Skinner}, Financial Markets Commentator}

\section{Project Summary}
In this project we set out to examine the pricing of traditional interest rate swaps and their extension to the world of cryptocurrencies. We discussed the simplicity of the pricing formulas for the traditional case and identified that the calculation merely consists of a time value of money problem for the associated cash flows. Further, we found that we could use spot rates, forward rates, or discount factors to price a plain vanilla fixed-for-floating interest rate swap––and that knowledge of one allowed us to compute each of the others through their interlinked nature. 

We then considered this premise from the opposite perspective, noting that it is market forces which determine the price of such swaps, hence, the primary challenge is constructing appropriate curves. We looked at two straightforward approaches to curve construction: explicit and implicit interpolation of the discount factors, and presented a simplified Python implementation using observed market data. Our results indicated the difficulty of such construction, highlighting the instability of the forward curve and the fact that the condition of no arbitrage, based only on a finite number of instruments, is not sufficient to produce smooth curves.

We concluded our study of the traditional case by discussing some of the practical considerations omitted from our analysis. This included the idea of day count convention, which is different depending on the geographical market for the interest rate swap itself. In addition, we noted the structural changes that have occurred since the financial crisis with the switch to multi-curve pricing––that is, using a combination of overnight indexed swaps and LIBOR-based swaps to produce forward curves––and more recently, financial markets' transition away from the LIBOR benchmark rate to alternative risk free rates.

The project then proceeded to explore the crypto market. We first analysed the concept of interest, stating that the owners of cryptocurrencies are not those that are rewarded, in contradiction to traditional markets. We reviewed traditional interest rate models and found that the short rate would be identically zero in the crypto case, questioning the existence of a non-trivial interest rate model.  However, we discussed a recent paper which proposed use of a Bessel(3) process which could satisfy the aforementioned condition whilst remaining non-trivial, before producing some indicative yield curves. 

We covered the pricing of one specific crypto interest rate swap: the BitMEX XBTUSD Funding Rate Swap. We found it had parallels to the traditional case, with a simplistic pricing formula––albeit based on a smaller timescale––but a slightly more complicated underlying in the funding rate. Our analysis was limited, however, by the lack of available historical data. Finally, we concluded by briefly mentioning the only other product on the market: the DAI Savings Rate Swap, and proposed some possible avenues which could theoretically host interest rate swap-like products. 

\section{Outlook and Future Work}

Overall, this project achieved its objective to provide an initial bridge between the extensive literature of traditional interest rate derivatives, specifically interest rate swaps, and the newer world of cryptocurrency derivatives. As such, it is likely of interest to both traditional and decentralised market participants seeking a concise introduction to the topic. Alternatively, this project could spark the interest of academics currently in the field of cryptocurrencies and digital finance who may wish to explore the topic in more detail.

As a result, there are plenty of potential avenues for further research across several levels. From an academic perspective, a deeper analysis and evaluation of different curve construction techniques, with computational implementation, would make an excellent undergraduate or postgraduate project for those programmes which have a finance and computer science component. We would recommend the student begin with a review of the context of curve construction, by which we mean the history and an understanding of the financial products to which it applies. This could be followed by researching the work of \cite{hagan2006interpolation}, who performed something similar, summarising the mathematical principles and implementing a variety of the methods in the preferred coding language.

Continuing at the undergraduate or postgraduate level, an analysis traditional interest rate swaps in a \textit{post-LIBOR} world could be another potential project. We recommend a thorough analysis of the alternative risk free rates across the different jurisdictions, their calculation methodologies, and an exploration of the changes to pricing interest rate swaps (and potentially other products related to benchmark rates).

As we saw in Chapter 4, there is little existing literature regarding cryptocurrency interest rates and specifically crypto interest rate swaps. This represents a clear gap for further research, this time at a more advanced academic level, into a mathematical model of crypto interest rates. Furthermore, the theoretical products mentioned in our project could be mathematically formalised and economically evaluated to understand the circumstances under which they could viable––which is clearly of benefit to industry partners. 

The final, and perhaps simplest, option could be to repeat and extend the analysis of this project with improved data. This could be in the form of more recent traditional market data for the pricing of interest rate swaps, and especially for curve construction, or using obtainable market data for cryptocurrencies. 